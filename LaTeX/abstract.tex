\documentclass[11pt]{article}
\makeatletter\if@twocolumn\PassOptionsToPackage{switch}{lineno}\else\fi\makeatother



\usepackage{titlesec}
\usepackage{amsfonts,amssymb,amsbsy,latexsym,amsmath,tabulary,graphicx,times,xcolor}
\usepackage[utf8x]{inputenc}
\usepackage{fancyhdr}
\def\NormalBaseline{\def\baselinestretch{1.1}}
\makeatletter
\def\hlinewd#1{%
  \noalign{\ifnum0=`}\fi\hrule \@height #1%
  \futurelet\reserved@a\@xhline}
\def\tbltoprule{\hlinewd{.8pt}}%\\[-10pt]}
\def\tblbottomrule{\hlinewd{.8pt}}
\def\tblmidrule{\hline\noalign{\vspace*{0pt}}}

\def\@shorttitle{\@empty}
\def\shorttitle#1{\gdef\@shorttitle{#1}}

\fancypagestyle{custom}{
\fancyhf{}
\fancyhead[C]{\@shorttitle}

\fancyfoot[C]{}
}
\fancypagestyle{plain}{
\fancyhf{}
}


\makeatother

\usepackage{times}

\usepackage[a4paper,margin=2cm,headsep=.9cm,headheight=18pt,top=2.5cm,footnotesep=1.5\baselineskip]{geometry}
\usepackage{caption}
\captionsetup[figure]{labelfont=bf,labelsep=newline,justification=centerlast,labelfont={small,sc,bf},font=small,aboveskip=.0\baselineskip}

\captionsetup[table]{labelfont=bf,labelsep=newline,justification=centerlast,labelfont={small,sc,bf},font=small,aboveskip=.3\baselineskip}
\linespread{1.5}

\setcounter{totalnumber}{4}
\def\topfraction{0.9}
\def\bottomfraction{0.4}
\def\floatpagefraction{0.8}
\def\textfraction{0.1}
\widowpenalty 10000
\clubpenalty 10000
\makeatletter
\setlength\intextsep   {1.5\baselineskip \@plus 2\p@ \@minus 2\p@}
\makeatother

  
%%%%%%%%%%%%%%%%%%%%%%%%%%%%%%%%%%%%%%%%%%%%%%%%%%%%%%%%%%%%%%%%%%%%%%%%%%
% Following additional macros are required to function some 
% functions which are not available in the class used.
%%%%%%%%%%%%%%%%%%%%%%%%%%%%%%%%%%%%%%%%%%%%%%%%%%%%%%%%%%%%%%%%%%%%%%%%%%
\usepackage{url,multirow,morefloats,floatflt,cancel}
\makeatletter


\AtBeginDocument{\@ifpackageloaded{textcomp}{}{\usepackage{textcomp}}}
\makeatother
\usepackage{colortbl}
\usepackage{xcolor}
\usepackage{pifont}
\usepackage[nointegrals]{wasysym}
\urlstyle{rm}
\makeatletter

%%%For Table column width calculation.
\def\mcWidth#1{\csname TY@F#1\endcsname+\tabcolsep}

%%Hacking center and right align for table
\def\cAlignHack{\rightskip\@flushglue\leftskip\@flushglue\parindent\z@\parfillskip\z@skip}
\def\rAlignHack{\rightskip\z@skip\leftskip\@flushglue \parindent\z@\parfillskip\z@skip}

%Etal definition in references
\@ifundefined{etal}{\def\etal{\textit{et~al}}}{}


%\if@twocolumn\usepackage{dblfloatfix}\fi
\usepackage{ifxetex}
\ifxetex\else\if@twocolumn\@ifpackageloaded{stfloats}{}{\usepackage{dblfloatfix}}\fi\fi

\AtBeginDocument{
\expandafter\ifx\csname eqalign\endcsname\relax
\def\eqalign#1{\null\vcenter{\def\\{\cr}\openup\jot\m@th
  \ialign{\strut$\displaystyle{##}$\hfil&$\displaystyle{{}##}$\hfil
      \crcr#1\crcr}}\,}
\fi
}

%For fixing hardfail when unicode letters appear inside table with endfloat
\AtBeginDocument{%
  \@ifpackageloaded{endfloat}%
   {\renewcommand\efloat@iwrite[1]{\immediate\expandafter\protected@write\csname efloat@post#1\endcsname{}}}{\newif\ifefloat@tables}%
}%

\def\BreakURLText#1{\@tfor\brk@tempa:=#1\do{\brk@tempa\hskip0pt}}
\let\lt=<
\let\gt=>
\def\processVert{\ifmmode|\else\textbar\fi}
\let\processvert\processVert

\@ifundefined{subparagraph}{
\def\subparagraph{\@startsection{paragraph}{5}{2\parindent}{0ex plus 0.1ex minus 0.1ex}%
{0ex}{\normalfont\small\itshape}}%
}{}

% These are now gobbled, so won't appear in the PDF.
\newcommand\role[1]{\unskip}
\newcommand\aucollab[1]{\unskip}
  
\@ifundefined{tsGraphicsScaleX}{\gdef\tsGraphicsScaleX{1}}{}
\@ifundefined{tsGraphicsScaleY}{\gdef\tsGraphicsScaleY{.9}}{}
% To automatically resize figures to fit inside the text area
\def\checkGraphicsWidth{\ifdim\Gin@nat@width>\linewidth
	\tsGraphicsScaleX\linewidth\else\Gin@nat@width\fi}

\def\checkGraphicsHeight{\ifdim\Gin@nat@height>.9\textheight
	\tsGraphicsScaleY\textheight\else\Gin@nat@height\fi}

\def\fixFloatSize#1{}%\@ifundefined{processdelayedfloats}{\setbox0=\hbox{\includegraphics{#1}}\ifnum\wd0<\columnwidth\relax\renewenvironment{figure*}{\begin{figure}}{\end{figure}}\fi}{}}
\let\ts@includegraphics\includegraphics

\def\inlinegraphic[#1]#2{{\edef\@tempa{#1}\edef\baseline@shift{\ifx\@tempa\@empty0\else#1\fi}\edef\tempZ{\the\numexpr(\numexpr(\baseline@shift*\f@size/100))}\protect\raisebox{\tempZ pt}{\ts@includegraphics{#2}}}}

%\renewcommand{\includegraphics}[1]{\ts@includegraphics[width=\checkGraphicsWidth]{#1}}
\AtBeginDocument{\def\includegraphics{\@ifnextchar[{\ts@includegraphics}{\ts@includegraphics[width=\checkGraphicsWidth,height=\checkGraphicsHeight,keepaspectratio]}}}

\DeclareMathAlphabet{\mathpzc}{OT1}{pzc}{m}{it}

\def\URL#1#2{\@ifundefined{href}{#2}{\href{#1}{#2}}}

%%For url break
\def\UrlOrds{\do\*\do\-\do\~\do\'\do\"\do\-}%
\g@addto@macro{\UrlBreaks}{\UrlOrds}



\edef\fntEncoding{\f@encoding}
\def\EUoneEnc{EU1}
\makeatother
\def\floatpagefraction{0.8} 
\def\dblfloatpagefraction{0.8}
\def\style#1#2{#2}
\def\xxxguillemotleft{\fontencoding{T1}\selectfont\guillemotleft}
\def\xxxguillemotright{\fontencoding{T1}\selectfont\guillemotright}

\newif\ifmultipleabstract\multipleabstractfalse%
\newenvironment{typesetAbstractGroup}{}{}%

%%%%%%%%%%%%%%%%%%%%%%%%%%%%%%%%%%%%%%%%%%%%%%%%%%%%%%%%%%%%%%%%%%%%%%%%%%

\usepackage{natbib}




\usepackage{titlesec}
\usepackage[T1]{fontenc}
\setcounter{secnumdepth}{5}
 
\titleformat{\section}[hang]{\NormalBaseline\filright\large\bfseries}
{\large\thesection}
{10pt}
{}
[]
\titleformat{\subsection}[hang]{\NormalBaseline\filright\bfseries}
{\thesubsection}
{10pt}
{}
[]
\titleformat{\subsubsection}[hang]{\NormalBaseline\filright\bfseries\itshape}
{\upshape\thesubsubsection}
{10pt}
{}
[]
\titleformat{\paragraph}[runin]{\NormalBaseline\filright\bfseries}
{\theparagraph}
{10pt}
{}
[]
\titleformat{\subparagraph}[runin]{\NormalBaseline\filright\bfseries\itshape}
{\thesubparagraph}
{10pt}
{}
[]

\titlespacing{\section}{0pt}{1.5\baselineskip}{.2\baselineskip}  
\titlespacing{\subsection}{0pt}{1.5\baselineskip}{.2\baselineskip}  
\titlespacing{\subsubsection}{0pt}{1.5\baselineskip}{.2\baselineskip}  
\titlespacing{\paragraph}{0pt}{.5\baselineskip}{10pt}  
\titlespacing{\subparagraph}{0pt}{.5\baselineskip}{10pt}  
  

  




\begin{document}


\renewcommand*\rmdefault{bch}\normalfont\upshape




\shorttitle{}

\date{}  

  
\title{\NormalBaseline\raggedright\bfseries \textbf{Content:}\\\vspace{.2cm}















\vspace{-3em}}  
      	\def\AuAffLabelStyle#1{\textsuperscript{\upshape#1}}
        \def\AuFont{\bfseries\large}
        \def\AuSep{, }
        \def\AffSep{\\}
        \let\origthanks\thanks
\renewcommand\thanks[1]{\begingroup\let\rlap\relax\origthanks{#1}\endgroup}
\author{\hskip2pc\parbox{.95\linewidth}{\AuFont 
    % Address
    }}
    
    
\maketitle 
\pagestyle{custom}


\LARGE
Declaration


Acknowledgment


Abstract


1) Vector and product of vectors


2) Scalar product of vectors


\hspace{1cm}2.1) Geometrical interpretation 


\hspace{2.2cm}2.1.1) Case: Perpendicular vectors


\hspace{2.2cm}2.1.2) Case: Co-directional vectors


\hspace{2.2cm}2.1.3) Angle between two vectors


\hspace{2.2cm}2.1.4) Length of a vector as scalar product


\hspace{2.2cm}2.1.5) Properties of scalar product


\hspace{3.8cm} Scalar projection


\hspace{3.8cm} Associative law


\hspace{3.8cm} Commutative law


\hspace{3.8cm} Distributive law



3) Observation


4) Conclusion


5) Bibliography/Refrences


\newpage

\section{Objective :}
 To write four compound statements by using two simple mathematical statements.
    
\section{Procedure/ Explanation :}
In any triangle ABC,


p : $a=b\cos C + c\cos B$


q : $b=c\cos A + a \cos C$


Here, p and q are two simple statements. Both of them are true mathematical statements regarding projection law in a triangle. The compound statements that can be formed using statements p and q are as follows:
\subsection{p $\wedge{}$ q}
$a=b\cos C+c \cos B$ AND $b=c \cos A+a \cos C$


Truth table:\\
\begin{center}
\begin{tabular}{ |p{2.2cm}||p{1.9cm}||p{3.3cm}|  }
 \hline
 p & q & p $\wedge$ q\\
 \hline
 T & T & T \\
 \hline
\end{tabular}
\end{center} 

\subsection{p $\vee$ q}
$a=b\cos C+c \cos B$ OR $b=c \cos A+a \cos C$


Truth table:\\
\begin{center}
\begin{tabular}{ |p{2.2cm}||p{1.9cm}||p{3.3cm}|  }
 \hline
 p & q & p $\vee$ q\\
 \hline
 T & T & T \\
 \hline
\end{tabular}
\end{center} 
\newpage

\subsection{$\backsim$p $\vee$ q}
$a\neq b\cos C+c \cos B$ OR $b=c \cos A+a \cos C$


Truth table:\\
\begin{center}
\begin{tabular}{ |p{2.2cm}||p{1.9cm}||p{3.3cm}|  }
 \hline
 $\backsim$p & q & $\backsim$p $\vee$ q\\
 \hline
 F & T & T \\
 \hline
\end{tabular}
\end{center} 
\subsection{p $\Rightarrow$ $\backsim$q}
IF $a=b\cos C+c \cos B$, THEN $b \neq c\cos A +a \cos C$


Truth table:\\
\begin{center}
\begin{tabular}{ |p{2.2cm}||p{1.9cm}||p{3.3cm}|  }
 \hline
 p & $\backsim$q & p $\Rightarrow$ $\backsim$q\\
 \hline
 T & F & F \\
 \hline
\end{tabular}
\end{center} 



\section{Observation:}
\mbox{For any two true simple statements p and q, the compound statements (p $\wedge$ q), (p $\vee$ q) and ($\backsim$p $\vee$ q)}are true and the statement (p $\Rightarrow$ $\backsim$q) is found to be false.
    
\section{Conclusion:}
Two simple mathematical statements can be used to form a number of compound statements by using the logical connectives conjuction ($\wedge$), disjunction ($\vee$), negation ($\backsim$), implication ($\Rightarrow$) and equivalence ($\Leftrightarrow$). The resultant statements may be either 'True' or 'False' depending upon the original simple statements.\\[7cm]
\hspace{-1cm}
................................................................................................................................................................

\bibliographystyle{blank}

\bibliography{\jobname}

\end{document}
