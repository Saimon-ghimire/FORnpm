\documentclass[11pt]{article}
\makeatletter\if@twocolumn\PassOptionsToPackage{switch}{lineno}\else\fi\makeatother



\usepackage{titlesec}
\usepackage{amsfonts,amssymb,amsbsy,latexsym,amsmath,tabulary,graphicx,times,xcolor}
\usepackage[utf8x]{inputenc}
\usepackage{fancyhdr}
\def\NormalBaseline{\def\baselinestretch{1.1}}
\makeatletter
\def\hlinewd#1{%
  \noalign{\ifnum0=`}\fi\hrule \@height #1%
  \futurelet\reserved@a\@xhline}
\def\tbltoprule{\hlinewd{.8pt}}%\\[-10pt]}
\def\tblbottomrule{\hlinewd{.8pt}}
\def\tblmidrule{\hline\noalign{\vspace*{0pt}}}

\def\@shorttitle{\@empty}
\def\shorttitle#1{\gdef\@shorttitle{#1}}

\fancypagestyle{custom}{
\fancyhf{}
\fancyhead[C]{\@shorttitle}

\fancyfoot[C]{}
}
\fancypagestyle{plain}{
\fancyhf{}
}


\makeatother

\usepackage{times}

\usepackage[a4paper,margin=2cm,headsep=.1cm,headheight=18pt,top=1.5cm,footnotesep=1.5\baselineskip]{geometry}
\usepackage{caption}
\captionsetup[figure]{labelfont=bf,labelsep=newline,justification=centerlast,labelfont={small,sc,bf},font=small,aboveskip=.0\baselineskip}

\captionsetup[table]{labelfont=bf,labelsep=newline,justification=centerlast,labelfont={small,sc,bf},font=small,aboveskip=.3\baselineskip}
\linespread{1.5}

\def\topfraction{0.9}
\def\bottomfraction{0.4}
\def\floatpagefraction{0.8}
\def\textfraction{0.1}
\widowpenalty 10000
\clubpenalty 10000
\makeatletter
\setlength\intextsep   {1.5\baselineskip \@plus 2\p@ \@minus 2\p@}
\makeatother

  
%%%%%%%%%%%%%%%%%%%%%%%%%%%%%%%%%%%%%%%%%%%%%%%%%%%%%%%%%%%%%%%%%%%%%%%%%%
% Following additional macros are required to function some 
% functions which are not available in the class used.
%%%%%%%%%%%%%%%%%%%%%%%%%%%%%%%%%%%%%%%%%%%%%%%%%%%%%%%%%%%%%%%%%%%%%%%%%%
\usepackage{url,multirow,morefloats,floatflt,cancel}
\makeatletter


\AtBeginDocument{\@ifpackageloaded{textcomp}{}{\usepackage{textcomp}}}
\makeatother
\usepackage{colortbl}
\usepackage{xcolor}
\usepackage{pifont}
\usepackage[nointegrals]{wasysym}
\urlstyle{rm}
\makeatletter

%%%For Table column width calculation.
\def\mcWidth#1{\csname TY@F#1\endcsname+\tabcolsep}

%%Hacking center and right align for table
\def\cAlignHack{\rightskip\@flushglue\leftskip\@flushglue\parindent\z@\parfillskip\z@skip}
\def\rAlignHack{\rightskip\z@skip\leftskip\@flushglue \parindent\z@\parfillskip\z@skip}

%Etal definition in references
\@ifundefined{etal}{\def\etal{\textit{et~al}}}{}


%\if@twocolumn\usepackage{dblfloatfix}\fi
\usepackage{ifxetex}
\ifxetex\else\if@twocolumn\@ifpackageloaded{stfloats}{}{\usepackage{dblfloatfix}}\fi\fi

\AtBeginDocument{
\expandafter\ifx\csname eqalign\endcsname\relax
\def\eqalign#1{\null\vcenter{\def\\{\cr}\openup\jot\m@th
  \ialign{\strut$\displaystyle{##}$\hfil&$\displaystyle{{}##}$\hfil
      \crcr#1\crcr}}\,}
\fi
}

%For fixing hardfail when unicode letters appear inside table with endfloat
\AtBeginDocument{%
  \@ifpackageloaded{endfloat}%
   {\renewcommand\efloat@iwrite[1]{\immediate\expandafter\protected@write\csname efloat@post#1\endcsname{}}}{\newif\ifefloat@tables}%
}%

\def\BreakURLText#1{\@tfor\brk@tempa:=#1\do{\brk@tempa\hskip0pt}}
\let\lt=<
\let\gt=>
\def\processVert{\ifmmode|\else\textbar\fi}
\let\processvert\processVert

\@ifundefined{subparagraph}{
\def\subparagraph{\@startsection{paragraph}{5}{2\parindent}{0ex plus 0.1ex minus 0.1ex}%
{0ex}{\normalfont\small\itshape}}%
}{}

% These are now gobbled, so won't appear in the PDF.
\newcommand\role[1]{\unskip}
\newcommand\aucollab[1]{\unskip}
  
\@ifundefined{tsGraphicsScaleX}{\gdef\tsGraphicsScaleX{1}}{}
\@ifundefined{tsGraphicsScaleY}{\gdef\tsGraphicsScaleY{.9}}{}
% To automatically resize figures to fit inside the text area
\def\checkGraphicsWidth{\ifdim\Gin@nat@width>\linewidth
	\tsGraphicsScaleX\linewidth\else\Gin@nat@width\fi}

\def\checkGraphicsHeight{\ifdim\Gin@nat@height>.9\textheight
	\tsGraphicsScaleY\textheight\else\Gin@nat@height\fi}

\def\fixFloatSize#1{}%\@ifundefined{processdelayedfloats}{\setbox0=\hbox{\includegraphics{#1}}\ifnum\wd0<\columnwidth\relax\renewenvironment{figure*}{\begin{figure}}{\end{figure}}\fi}{}}
\let\ts@includegraphics\includegraphics

\def\inlinegraphic[#1]#2{{\edef\@tempa{#1}\edef\baseline@shift{\ifx\@tempa\@empty0\else#1\fi}\edef\tempZ{\the\numexpr(\numexpr(\baseline@shift*\f@size/100))}\protect\raisebox{\tempZ pt}{\ts@includegraphics{#2}}}}

%\renewcommand{\includegraphics}[1]{\ts@includegraphics[width=\checkGraphicsWidth]{#1}}
\AtBeginDocument{\def\includegraphics{\@ifnextchar[{\ts@includegraphics}{\ts@includegraphics[width=\checkGraphicsWidth,height=\checkGraphicsHeight,keepaspectratio]}}}

\DeclareMathAlphabet{\mathpzc}{OT1}{pzc}{m}{it}

\def\URL#1#2{\@ifundefined{href}{#2}{\href{#1}{#2}}}

%%For url break
\def\UrlOrds{\do\*\do\-\do\~\do\'\do\"\do\-}%
\g@addto@macro{\UrlBreaks}{\UrlOrds}



\edef\fntEncoding{\f@encoding}
\def\EUoneEnc{EU1}
\makeatother
\def\floatpagefraction{0.8} 
\def\dblfloatpagefraction{0.8}
\def\style#1#2{#2}
\def\xxxguillemotleft{\fontencoding{T1}\selectfont\guillemotleft}
\def\xxxguillemotright{\fontencoding{T1}\selectfont\guillemotright}

\newif\ifmultipleabstract\multipleabstractfalse%
\newenvironment{typesetAbstractGroup}{}{}%

%%%%%%%%%%%%%%%%%%%%%%%%%%%%%%%%%%%%%%%%%%%%%%%%%%%%%%%%%%%%%%%%%%%%%%%%%%

\usepackage{natbib}




\usepackage{titlesec}
\usepackage[T1]{fontenc}

 
\titleformat{\section}[hang]{\NormalBaseline\filright\large\bfseries}
{\large\thesection}
{10pt}
{}
[]
\titleformat{\subsection}[hang]{\NormalBaseline\filright\bfseries}
{\thesubsection}
{10pt}
{}
[]
\titleformat{\subsubsection}[hang]{\NormalBaseline\filright\bfseries\itshape}
{\upshape\thesubsubsection}
{10pt}
{}
[]
\titleformat{\paragraph}[runin]{\NormalBaseline\filright\bfseries}
{\theparagraph}
{10pt}
{}
[]
\titleformat{\subparagraph}[runin]{\NormalBaseline\filright\bfseries\itshape}
{\thesubparagraph}
{10pt}
{}
[]

\titlespacing{\section}{0pt}{1.5\baselineskip}{.2\baselineskip}  
\titlespacing{\subsection}{0pt}{1.5\baselineskip}{.2\baselineskip}  
\titlespacing{\subsubsection}{0pt}{1.5\baselineskip}{.2\baselineskip}  
\titlespacing{\paragraph}{0pt}{.5\baselineskip}{10pt}  
\titlespacing{\subparagraph}{0pt}{.5\baselineskip}{10pt}  
  

  




\begin{document}


\renewcommand*\rmdefault{bch}\normalfont\upshape




\shorttitle{}

\date{}  

  
\title{\NormalBaseline\raggedright\bfseries \textbf{Problem:} Following are the marks obtained by the students X and Y in 6 tests of 100 marks each. If the consistency of the performance is the criteria for awarding a prize, who should get the prize?\\ \vspace{.5cm}
\begin{tabular}{ |p{1.5cm}||p{1.5cm}|p{1.5cm}|p{1.5cm}|p{1.5cm}|p{1.5cm}|p{1.5cm}|   }
 \hline
  \textbf{Test} & \textit{1} & \textit{2} & \textit{3} & \textit{4} & \textit{5} & \textit{6} \\
 \hline
 \textbf{X} & 56 & 72 & 48 & 69 & 64 & 81 \\
 \textbf{Y} & 63 & 74 & 45 & 57 & 82 & 63 \\
 \hline
\end{tabular}
\centering















\vspace{-3em}}
  
      	\def\AuAffLabelStyle#1{\textsuperscript{\upshape#1}}
        \def\AuFont{\bfseries\large}
        \def\AuSep{, }
        \def\AffSep{\\}
        \let\origthanks\thanks
\renewcommand\thanks[1]{\begingroup\let\rlap\relax\origthanks{#1}\endgroup}
\author{\hskip2pc\parbox{.95\linewidth}{\AuFont 
    % Address
    }}
    
    
\maketitle 
\pagestyle{custom}

    
\section{Objective :}
 To find variance in the given distribution.
    
\section{Procedure/ Explanation :}
 Tabulating the data for X:\\
 \vspace{-0.5cm}
 \begin{center}
 \begin{tabular}{ |p{2.2cm}||p{1.9cm}||p{3.3cm}|  }
 \hline
 $X$ & $X- \bar{X}_x$ & $(X-\bar{X}_x)^2$\\
 \hline
 48 & -17 & 289 \\
 56 & -9 & 81  \\
 64 & -1 & 1 \\
 69 & 4 & 16 \\
 72 & 7 & 49 \\
 81 & 16 & 256 \\
 \hline
 $ \sum X=390$ &  & $ \sum (X-\bar{X}_x)^2=692$ \\
 \hline
\end{tabular}
\end{center}
\vspace{0.5cm}

N=6


$\bar{X}_x=\displaystyle\frac{\sum X}{N}=\dfrac{390}{6}=65$\\[5pt]


$\sigma_x= \sqrt{\dfrac{\sum (X-\bar{X}_x)^2}{N}}
= \sqrt{\dfrac{692}{6}}=10.74 $\\[5pt]

Coefficient of variation $(C.V_x)=\dfrac{\sigma_x}{\bar{X}_x}*100 = \dfrac{10.74}{65}*100 = 16.52$\\[5cm]

Tabluating the data for Y:
\vspace{0.3cm}
 \begin{center}
 \begin{tabular}{ |p{2.2cm}||p{1.9cm}||p{2.3cm}||p{3.4cm}|  }
 \hline
 $X$ & $f$ & $fX$ & $f(X-\bar{X}_y)^2$ \\
 \hline
 45 & 1 & 45 & 361 \\
 57 & 1 & 57 & 49 \\
 63 & 2 & 126 & 2 \\
 74 & 1 & 74 & 100 \\
 82 & 1 & 82 & 324 \\
 \hline
  &   & $\sum fX=384$&$ \sum f(X-\bar{X}_y)^2=836$  \\
 \hline
\end{tabular}
\end{center}
\vspace{0.5cm}

N=6


$\bar{X}_y=\displaystyle\frac{\sum fX}{N}=\dfrac{384}{6}=64$\\[5pt]


$\sigma_y= \sqrt{\dfrac{\sum (X-\bar{X}_y)^2}{N}}
= \sqrt{\dfrac{835}{6}}=11.80 $\\[5pt]

Coefficient of variation $(C.V_y)=\dfrac{\sigma_y}{\bar{X}_y}*100 = \dfrac{11.80}{64}*100 = 18.44$\\[1cm]


$\therefore$ $C.V_x<C.V_y.$\hspace{.2cm} Hence,   data in set $Y$ has more variation.


































    

\section{Observation:}
The coefficient of variation of the marks of student X in less than that of student Y. Thus, \mbox{student X} has less variant (i.e, more consistent) marks.
    
\section{Conclusion:}
Uniformity of data in a given distribution can be determined with the help of co-efficient of \mbox{variation} of the data. The higher the co-efficient of variation, the greater the level of dispersion around the mean (or any arbitiary number taken).\\[4.3cm]
\hspace{-1cm}
................................................................................................................................................................

\bibliographystyle{blank}

\bibliography{\jobname}

\end{document}
